\documentclass[a4paper]{article}
\usepackage[utf8]{inputenc}
\usepackage[russian]{babel}
\usepackage{listings}
\usepackage[a4paper]{geometry}
\usepackage{indentfirst}
\usepackage{graphicx}
\usepackage{caption}
\usepackage{float}
\usepackage{amssymb}
\usepackage{url}
\usepackage{physics}

\begin{document}

\title{Отчёт по курсу <<Методы решения наукоемких большеразмерных прикладных труднорешаемых задач>>.}
\author{Владислав Соврасов\\ аспирант гр. 2-о-051318}
\date{}
\maketitle

\section{Сравнение различных сратегий ветвления}

В ходе выполнение работы был реализован метод ветвей и границ для решения задачи о доставке заказов.
Данная задача имеет $n !$ допустимых решений, лишь некоторые из которых являются оптимальными
($n$ --- количество пунктов доставки).
Реализация поддерживает следующие стратегии ветвления: поиск в ширину, поиск в глубину,
оптимистичная, реалистичная.

В таблице \ref{tab:results} приведены показатели эффективности метода на тестовых задачах с различными стратегиями ветвления.
Для задачи о доставке эффективность вычисляется следующим способом: $T=1 - \frac{N_{vis}}{n!}$, где $n$ --- размерность
задачи, $N_{vis}$ --- количество обработанных методом ветвей и границ вершин.

На рассматриваемом наборе тестовых задач в среднем самым эффективным оказался метод с реалистичной стратегией
ветвления. Наихудший результат показал метод, использующий стратегию поиска в ширину.
В некоторых случаях решение, полученное жадным алгоритмом при построении верхней оценки оказывалось оптимальным
и методу ветвей и границ требовалось обработать всего одну вершину, чтобы решить задачу. Остальные вершины при этом
были сразу же отсечены.


\begin{table}[H]
  \caption{Эффективность и количество обработанных узлов при различных стретегиях ветвления}
  \label{tab:results}
\begin{tabular}{lllll}
\hline
 Problem   & depth-first   & optimistic & realistic  & breadth-first                 \\
\hline
 01        & 0.83333333333(1)    & 0.83333333333(1)    & 0.83333333333(1)    & 0.83333333333(1)     \\
 02        & 0.83333333333(1)    & 0.83333333333(1)    & 0.83333333333(1)    & 0.83333333333(1)     \\
 03        & 0.99997712743(83)   & \textbf{0.99998126102(68)}   & \textbf{0.99998126102(68)}   & 0.99997409612(94)    \\
 04        & 0.99998346561(60)   & \textbf{0.99998511905(54)}   & 0.9999845679(56)    & 0.99997161596(103)   \\
 05        & 0.99998015873(72)   & 0.9999845679(56)    & \textbf{0.99998897707(40)}   & 0.99997712743(83)    \\
 06        & 0.99999999952(628)  & 0.99999999996(49)   & \textbf{0.99999999996(48)}   & 0.99999999557(5787)  \\
 07        & 0.99999999551(5869) & 0.99999999759(3145) & \textbf{0.99999999911(1167)} & 0.99999993155(89508) \\
 08        & 1.0(87742)          & 1.0(80445)          & \textbf{1.0(80442)}          & 1.0(80457)           \\
 09        & 1.0(1)              & 1.0(1)              & 1.0(1)              & 1.0(1)               \\
 10        & 1.0(134208)         & \textbf{1.0(25935)}          & 1.0(145452)         & 1.0(32471)           \\
 \hline
 & \\
 \hline
 $T_{avg}$ & 0.96666074135       & 0.96666176122       & 0.96666214717       & 0.96665894333        \\
\hline
\end{tabular}
\end{table}

\section{Исходный код}
Полная версия исходного кода доступна в репозитории \url{https://github.com/sovrasov/branch-and-bounds-lab}
\lstinputlisting[language=Python, numbers=left]{../transportation_solver.py}

\end{document}
